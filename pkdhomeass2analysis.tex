\documentclass[12pt, a4paper]{article}
\usepackage[utf8]{inputenc}
\usepackage[swedish]{babel}
\usepackage{fullpage}
\usepackage{hyperref}
\usepackage{amsmath}
\usepackage{amssymb}
\usepackage{mathtools}
\usepackage[parfill]{parskip}

\title{Analysis of rectangles and quadtrees}
\author{Erik Englund, 921016-4530 and Alexander Andersson, 860616-1530}

\begin{document}
\maketitle

\section{Time complexity of validRectangle, emptyQtree, insert and query}

\subsection{validRectangle}

Since the validRectangle function only computes two comparisons, the time-complexity of the function is 
$\Theta(1)$




\subsection{emptyQtree}

This function only produces an empty quadtree with the extent of rectangle e. Thus the time-complexity of the function is constant $\Theta(1)$



\subsection{insert}

The insert function use constant time processes like comparisons and adding a head to a list (only one element, the rectangle that is to be inserted). This is the case throughout the whole function, including the recursive calls. Assuming that the root of the tree has a constant extent $n$, we can produce the following time-complexity formula:

$$T(n)=T\left(\frac{n}{4}\right)+\Theta(1)$$

Here we can apply the master theorem and the result is that $T(n)=\Theta(n)$. Since $n$ is constant, the time complexity for insert is therefore also constant $T(n) = \Theta(1)$.



\subsection{query}


Let $k$ be the depth of the tree. In this case $k$ is constant since depth is dependant on the extent, which is constant (given from assignment). Let $n$ be the total number of rectangles and $n_i$ the number of rectangles at depth $i$.
The query function uses comparisons which take constant time. It also uses list concatenation of all the rectangle lists on each level in the recursive calls. In worst case, we will find all rectangles in the whole tree $n$ along the path we move down the tree. As the concationation at level $i$ is linear to the number of rectangles at level $i$, the time complexity is as follows:

$$T(i) = T(i-1) + \Theta(n_i)$$

Because the depth of the tree $k$ is constant, the time complexity is therefore the sum of all rectangles on levels from $k$ to $0$.

$$T(i) = \Theta(\sum_{i=0}^kn_i) = \Theta(n) $$

Therefore, in worst case the time complexity is linear to the number of rectangles in the tree $n$.

\section{query called with a complete and symmetrical tree}

When every level of the quadtree contains the exact same amount of rectangles, the concatenation of the lists will take constant time in every recursive call giving us $\Theta(1)$ time-complexity on every level. This means that with the size of the extent as $n$, and with the size of the extent being divided in 4 by every recursion, the time complexity form will be the following:

 $$T(n)=\left(\frac{n}{4}\right)+\Theta(1)$$

which by the master theorem gives us:

 $$T(n)=\Theta(\log n)$$

But, as the size of the extent $n$ is a constant, as said in the assignment, this gives us:

 $$T(n)=\Theta(\log 1)=\Theta(1)$$

constant time!

\end{document}
